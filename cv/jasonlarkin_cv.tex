\documentclass{article}
\usepackage{fullpage}
\usepackage{amsmath}
\usepackage{amssymb}
\usepackage{hyperref}
\usepackage{helvet}
\usepackage[margin=0.75in]{geometry}
\renewcommand{\familydefault}{\sfdefault}
\textheight=10in
\pagestyle{empty}
%\raggedbottom
\raggedright

%  \renewcommand{\encodingdefault}{cg}
  %\renewcommand{\rmdefault}{lgrcmr}

\def\bull{\vrule height 0.8ex width .7ex depth -.1ex }
% DEFINITIONS FOR RESUME
\newcommand{\area}[2]{\vspace*{-9pt} \begin{verse}\textbf{#1}   #2 \end{verse}  }
\newcommand{\lineunder}{\vspace*{-8pt} \\ \hspace*{-18pt} \hrulefill \\}
\newcommand{\header}[1]{{\hspace*{-15pt}\vspace*{6pt} \textsc{#1}} \vspace*{-6pt} \lineunder}
\newcommand{\employer}[3]{{ \textbf{#1} (#2) \underline{\textbf{\emph{#3}}}\\  }}
\newcommand{\contact}[3]{
\vspace*{-8pt}
\begin{center}
{\LARGE \scshape {#1}}\\
#2 \lineunder 
#3
\end{center}
\vspace*{-8pt}
}
\newenvironment{achievements}{\begin{list}{$\bullet$}{\topsep 0pt \itemsep -2pt}}{\vspace*{4pt}\end{list}}
\newcommand{\schoolwithcourses}[3]{
 \textbf{#1} #2 $\bullet$ #3\\ 
\vspace*{5pt}
}
\newcommand{\school}[3]{
 \textbf{#1} #2 $\bullet$ #3\\
}
% END RESUME DEFINITIONS

\begin{document}

\small
\smallskip
\vspace*{-44pt}

\contact{\href{http://jasonlarkin.github.io}{Jason M Larkin}}
{4763 Sherwood Dr, Pittsburgh PA 15236 $\bullet$ (412) 398-8813 $\bullet$ \href{mailto:jasonlarkin84@gmail.com}{jasonlarkin84@gmail.com} $\bullet$ 
\href{http://jasonlarkin.github.io}{http://jasonlarkin.github.io} }

\header{Career Overview and Objective}
I have extensive experience performing experimental and numerical studies in condensed matter physics. My interests include multi-language development for applications ranging from high-performance parallel computing to smart phones, and open-source collaboration to improve the way research is performed and the way results are disseminated.

\header{Education}

\schoolwithcourses{Carnegie Mellon University}{Pittsburgh, PA}{PhD Mechanical Engineering, 2013 GPA: 3.85/4.00}

\area{
Thesis: \href{http://jasonlarkin.github.io/projects-phd.html}
{Thermal Modeling of Disordered Materials.}}
{Numerically investigated thermal properties of crystal alloys, glasses, and organic materials using classical and \emph{ab initio}-based atomistic techniques.} 

\area{Coursework:}
{
molecular/electron structure $\cdot$ nanoscale transport phenomena $\cdot$ numerical methods
}

\schoolwithcourses{University of Pittsburgh}{Pittsburgh, PA}{MS Mechanical Engineering, 2009 GPA: 3.70/4.00}

\area{
Thesis: \href{http://jasonlarkin.github.io/projects-ms.html}
{Statistics of Particle Concentrations in Free-Surface Turbulence.}}
{Performed experiments using novel 2D and 3D flow configurations to study turbulence 
as a nonlinear dynamical system.}

\area{Coursework:}
{
quantum and statistical physics $\cdot$ turbulence $\cdot$ chaos and nonlinear phenomena
}

\schoolwithcourses{University of Pittsburgh}{Pittsburgh, PA}{BS Mechanical Engineering, 2007 GPA: 3.20/4.00}

\area{
Research: }
{Used computational fluid dynamics to design a model arterial bifurcation for \emph{in vivo} study.}

\header{Experience}

\employer{Carnegie Mellon University}{2010-2012}{Teaching Assistant-Heat Transfer}
	\begin{achievements}
	\item Topics in conduction, convection, and radiation. Supervised recitations and substituted for lectures. 
	\end{achievements}

\employer{University of Pittsburgh}{2008}{Teaching Assistant-Advanced Fluid Mechanics}
	\begin{achievements}
	\item Topics in viscous flow, boundary layer theory, and scale similarity. 
	\end{achievements}

\employer{University of Pittsburgh}{2007-2009}{Lecturer-Physics}
	\begin{achievements}
	\item Lectured to students and faculty on mathematics, bio-physics, turbulence, statistical and nonlinear phenomena. 
	\end{achievements}

\employer{Precision Therapeutics}{2006-2007}{Intern-Technology Development}
	\begin{achievements}
	\item Worked with team of software developers and laboratory equipment specialists.
	\item Used CADD to design and fabricate components of optical microscopes and laboratory automation controls. 
	\end{achievements}

\header{Skills}
\begin{achievements}

\item \textbf{Computing Languages:} Matlab, Fortran, Python, C/C++, Java, \LaTeX, Shell, Perl, Markdown, HTML.

\item \textbf{High-Performance Computing:} linux/unix cluster administration/computing, parallel computation (MPI, OpenMP), mixed-language development, open-source development (Git, \href{http://github.com/jasonlarkin}{Github}, \href{http://arxiv.org/find/physics/1/au:+Larkin_J/0/1/0/all/0/1}{arXiv}).

\item \textbf{General Computing:} linux/unix, Mac OS, Windows, Microsoft Office.

\item \textbf{Modeling:} atomistic simulation, quantum chemistry, nanoscale transport, statistical and nonlinear systems.

\item \textbf{Hardware:} general computing, optics/lasers, DI/DO AI/AO interfaces, simple automation, machining, circuitry. 
\end{achievements}

\header{Projects}
\begin{achievements}

\item \textbf{\href{http://ntpl.me.cmu.edu/research.html}{Quantum Mechanics-Driven Prediction of Nanostructure Thermal Conductivity}:}
served as investigator under the 
\href{http://www.wpafb.af.mil/afrl/afosr/}{AFOSR} with collaborators at Carnegie Mellon and Univ. of Pitt., performing 
calculations on the \href{http://www.hpcmo.hpc.mil/cms2/index.php}{DOD's HPCMP}.

\item \textbf{\href{https://github.com/jasonlarkin/disorder}{disorder}:} a comprehensive repository of open-source code and data from my PhD thesis, hosted on Github.

\item \textbf{\href{https://github.com/ntpl/ntpy}{ntpy}:} created this open-source  collaborative effort between members of \href{http://ntpl.me.cmu.edu/}{NTPL} and \href{http://www.mie.utoronto.ca/labs/atoms/}{University of Toronto}.

\item \textbf{\href{http://projects.ivec.org/gulp/}{GULP}:} international collaboration with \href{http://nanochemistry.curtin.edu.au/people/staff.cfm/J.Gale}{Julian Gale} at the 
\href{http://nanochemistry.curtin.edu.au/}{Nanochemistry Research Institute} at \href{http://www.curtin.edu.au/}{Curtin University}.

\item \textbf{\href{http://jasonlarkin.github.io/projects-ms.html}{Statistics of Free-Surface Turbulence}:} international collaboration with \href{http://perso.ens-lyon.fr/alain.pumir/Pumir_webpage.html}{Alain Pumir} at \href{http://www.ens-lyon.eu/annuaire/m-pumir-alain-83656.kjsp?RH=ZYZYZYZYZYZYZYZYZYZYZY}{ENS Lyon}, France.

\end{achievements}

\header{\href{http://jasonlarkin.github.io/pub.html}{Publications} (selected, 11 total)}
\begin{achievements}
\item "Origin of the Exceptionally Low Thermal Conductivity of Fullerene Derivative  PCBM Films", (in progress).
\item "Decorrelating a Compressible Turbulent Flow: an Experiment", \href{http://pre.aps.org/abstract/PRE/v82/i1/e016301}{Phys. Rev. E 82, 016301 (2010).}
\end{achievements}

\header{\href{http://jasonlarkin.github.io/pres.html}{Presentations} (selected, 15 total)}
\begin{achievements}
\item "Evaluation of the Virtual Crystal Approximation for Predicting Thermal Conductivity", J.M. Larkin (speaker), A.J.H.
   McGaughey, \href{http://www.mrs.org/spring2013/}{2013 MRS Spring Meeting} San Francisco, CA.
\item "The Generalized Fractal Dimensions of a 2-D Compressible Turbulence", J. Larkin (speaker), M. Bandi, W. Goldburg, \href{http://meetings.aps.org/Meeting/MAR08/Content/1017}{2008 American Physical Society March Meeting} New Orleans, LA.
\end{achievements}


\header{Honors}
\begin{achievements}
\item \href{http://www.asmeconferences.org/HT2012/}{2012 ASME MHNMT International Summer Heat Transfer Conference} Top 5 Technical Paper
\item \href{http://www.cmu.edu/me/news/archive/2011/bennett-conference.html}{2011 Bennett Conference Best Presentation}
\item \href{http://www.ices.cmu.edu/newsitem.asp?NewsID=749}{2011 ICES Northrop-Gruman Fellow}
\item 2007-2009 NSF Graduate Student Research Grant University of Pittsburgh Deptartment of Physics.
\end{achievements}

\header{Memberships}
\begin{achievements}
\item American Physical Society $\cdot$ American Society of Mechanical Engineers 
$\cdot$ Materials Research Society $\cdot$ Society of Industrial and Applied Mathematics $\cdot$ DOD High Performance Computing Modernization Program
\end{achievements}


\end{document}
