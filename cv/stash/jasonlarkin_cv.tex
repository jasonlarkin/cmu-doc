\documentclass{article}
\usepackage{fullpage}
\usepackage{amsmath}
\usepackage{amssymb}
\usepackage[dvipdfm,hidelinks,colorlinks=true]{hyperref}
\usepackage{helvet}
\usepackage[margin=0.75in]{geometry}
\renewcommand{\familydefault}{\sfdefault}
\textheight=10in
\pagestyle{empty}
%\raggedbottom
\raggedright

%  \renewcommand{\encodingdefault}{cg}
  %\renewcommand{\rmdefault}{lgrcmr}

\def\bull{\vrule height 0.8ex width .7ex depth -.1ex }
% DEFINITIONS FOR RESUME
\newcommand{\area}[2]{\vspace*{-9pt} \begin{verse}\textbf{#1}   #2 \end{verse}  }
\newcommand{\lineunder}{\vspace*{-8pt} \\ \hspace*{-18pt} \hrulefill \\}
\newcommand{\header}[1]{{\hspace*{-15pt}\vspace*{6pt} \textsc{#1}} \vspace*{-6pt} \lineunder}
\newcommand{\employer}[3]{{ \textbf{#1} (#2) \underline{\textbf{\emph{#3}}}\\  }}
\newcommand{\contact}[3]{
\vspace*{-8pt}
\begin{center}
{\LARGE \scshape {#1}}\\
#2 \lineunder 
#3
\end{center}
\vspace*{-8pt}
}
\newenvironment{achievements}{\begin{list}{$\bullet$}{\topsep 0pt \itemsep -2pt}}{\vspace*{4pt}\end{list}}
\newcommand{\schoolwithcourses}[3]{
 \textbf{#1} #2 $\bullet$ #3\\ 
\vspace*{5pt}
}
\newcommand{\school}[3]{
 \textbf{#1} #2 $\bullet$ #3\\
}
% END RESUME DEFINITIONS

\begin{document}

\small
\smallskip
\vspace*{-44pt}

\contact{\href{http://jasonlarkin.github.io}{Jason M Larkin}}
{42 Lintel Dr, McMurray PA 15317 $\bullet$ (412) 398-8813 $\bullet$ \href{mailto:jasonlarkin84@gmail.com}{jasonlarkin84@gmail.com} $\bullet$ 
\href{http://jasonlarkin.github.io}{http://jasonlarkin.github.io} }

\header{Career Overview and Objective}
I have extensive experience performing experimental and numerical studies in condensed matter physics with an emphasis on nanoscale transport, fluid dynamics, and nonlinear analysis. My interests include complex systems modeling, multi-language development for applications in high-performance parallel computing, and open-source collaboration to improve the way research is performed and results are disseminated.

\header{Education}

\schoolwithcourses{Carnegie Mellon University}{Pittsburgh, PA}{PhD Mechanical Engineering, 2013 GPA: 3.85/4.00}

\area{
Thesis: \href{http://jasonlarkin.github.io/projects-phd.html}
{Thermal Modeling of Disordered Materials.}}
{Numerically investigated thermal properties of crystal alloys, glasses, and organic materials using classical and \href{http://en.wikipedia.org/wiki/Ab_initio_quantum_chemistry_methods}{\emph{ab initio}}-based atomistic techniques.} 

\area{Coursework:}
{
molecular/electron structure $\cdot$ nanoscale transport phenomena $\cdot$ numerical methods
}

\schoolwithcourses{University of Pittsburgh}{Pittsburgh, PA}{MS Mechanical Engineering, 2009 GPA: 3.70/4.00}

\area{
Thesis: \href{http://jasonlarkin.github.io/projects-ms.html}
{Statistics of Particle Concentrations in Free-Surface Turbulence.}}
{Performed experiments using novel 2D and 3D flow configurations to study turbulence 
as a nonlinear dynamical system.}

\area{Coursework:}
{
quantum and statistical physics $\cdot$ turbulence $\cdot$ chaos and nonlinear phenomena
}

\schoolwithcourses{University of Pittsburgh}{Pittsburgh, PA}{BS Mechanical Engineering, 2007 GPA: 3.20/4.00}

\area{
Research: }
{Used \href{http://en.wikipedia.org/wiki/Finite_element_method}{FEM} to design a model arterial bifurcation for \emph{in vivo} study.}

\schoolwithcourses{Steel Center AVTS}{Jefferson Hills, PA}{\href{http://en.wikipedia.org/wiki/Computer-aided_design}{CADD} Certification, 2002 GPA: 3.80/4.00}
\area{
Coursework: }
{Trained in \href{http://en.wikipedia.org/wiki/Computer-aided_design}{CAD} using \href{http://www.autodesk.com/}{Autodesk}'s \href{http://www.autodesk.com/products/autocad/overview}{AutoCAD} (15.6) and \href{http://www.autodesk.com/products/autodesk-inventor-family/overview}{Inventor} (5.3) 
to produce machined products by \href{http://en.wikipedia.org/wiki/Computer-aided_manufacturing}{CAM} and human machining.  }

\header{Experience}

\employer{\href{http://spiralgen.com/}{SpiralGen, Inc.}}{2013-2014}{Software Engineer}

\textbf{\href{http://spiral.net/}{Spiral}}: tool to create automatically optimized/platform-tuned digital signal processing and numerical kernels.

%\header{Projects}
\begin{achievements}

\item Used integration work for \href{http://www.darpa.mil/default.aspx}{DARPA} projects (\href{http://www.darpa.mil/Our_Work/I2O/Programs/High-Assurance_Cyber_Military_Systems_(HACMS).aspx}{HACMS}, \href{http://www.darpa.mil/Our_Work/MTO/Programs/Power_Efficiency_Revolution_for_Embedded_Computing_Technologies_(PERFECT).aspx}{PERFECT}) to help create commercial version of the \href{http://spiral.net/}{Spiral} tool. 

- Designed and maintained \href{http://jenkins-ci.org/}{Jenkins} 
\href{http://en.wikipedia.org/wiki/Continuous_integration}{CI}  
environment for build/testing.

- Helped develop \href{http://nsis.sourceforge.net/Main_Page}{NSIS} installer for \href{http://spiral.net/}{Spiral} tool and integration with \href{http://jenkins-ci.org/}{Jenkins} 
\href{http://en.wikipedia.org/wiki/Continuous_integration}{CI}.

- Contributed static and context-sensitive help to \href{http://spiral.net/}{Spiral}'s 
\href{http://www.eclipse.org/}{Eclipse} plug-in.

- Created  native (C/C++/Fortran) and interpreted (\href{http://cython.org/}{Cython} and \href{http://www.mathworks.com/help/matlab/ref/mex.html}{Mex}) example implementations of \href{http://spiral.net/}{Spiral}-generated code. 

- Helped develop \href{https://wiki.hh.se/wg211/images/e/e0/M13Franchetti.pdf}{HCOL} implementations of robot control 
kernels (\href{http://en.wikipedia.org/wiki/PID_controller}{PID}, 
\href{http://en.wikipedia.org/wiki/Euler_method}{Euler}) for \href{http://www.darpa.mil/Our_Work/I2O/Programs/High-Assurance_Cyber_Military_Systems_(HACMS).aspx}{HACMS} project.  

\item \textbf{\href{http://www.darpa.mil/Our_Work/I2O/Programs/High-Assurance_Cyber_Military_Systems_(HACMS).aspx}{HACMS}:}
\href{http://www.darpa.mil/default.aspx}{DARPA} project to create technology for the construction of high-assurance cyber-physical systems. 

- Integrated and tested \href{http://spiral.net/}{Spiral}-generated, high-assurance code with simulated \href{http://www.cyberbotics.com/}{Webots} and physical \href{http://www.blackirobotics.com/}{Black-i Landshark} robots using \href{http://www.ros.org/}{ROS}. 

- Developed integrated \href{http://www.darpa.mil/NewsEvents/Releases/2014/05/21.aspx}{Pentagon demo} for HACMS using \href{http://www.ros.org/}{ROS}, \href{http://www.cyberbotics.com/}{Webots}, and \href{https://wiki.python.org/moin/TkInter}{TkInter}. 

- Developed an automated and virtualized \href{http://www.ros.org/}{ROS} test system for the Spiral-generated robot controller kernels.  

- Collaborated with a large/diverse team of labs (\href{http://www.hrl.com/}{HRL}, \href{http://www.sri.com/}{SRI}) and universities (\href{http://www.cmu.edu/}{CMU}, \href{http://web.mit.edu/}{MIT}, \href{http://www.princeton.edu/}{Princeton}, \href{http://illinois.edu/}{UIUC}, \href{http://www.upenn.edu/}{UPenn}). 

\item \textbf{\href{http://www.darpa.mil/Our_Work/MTO/Programs/Power_Efficiency_Revolution_for_Embedded_Computing_Technologies_(PERFECT).aspx}{PERFECT}:}
\href{http://www.darpa.mil/default.aspx}{DARPA} project to seek revolutionary approaches and to research and develop the technologies and techniques to provide the power efficiency required to enable embedded computing systems. 

- Helped develop an installable version of the \href{http://spiral.net/}{Spiral} tool with an \href{http://www.eclipse.org/}{Eclipse} plug-in to generate platform-optimized \href{http://en.wikipedia.org/wiki/Fast_Fourier_transform}{FFTs}. 
\end{achievements}

\employer{Carnegie Mellon University}{2010-2012}{Teaching Assistant-Heat Transfer}
	\begin{achievements}
	\item Topics in conduction, convection, and radiation. Supervised recitations and substituted for lectures. 
	\end{achievements}

\employer{University of Pittsburgh}{2008}{Teaching Assistant-Advanced Fluid Mechanics}
	\begin{achievements}
	\item Topics in viscous flow, boundary layer theory, and scale similarity. 
	\end{achievements}

\employer{University of Pittsburgh}{2007-2009}{Lecturer-Physics}
	\begin{achievements}
	\item Lectured to students and faculty on mathematics, bio-physics, turbulence, statistical and nonlinear phenomena. 
	\end{achievements}

\employer{Precision Therapeutics}{2006-2007}{Intern-Technology Development}
	\begin{achievements}
	\item Worked with team of software developers and laboratory equipment specialists.
	\item Used \href{http://en.wikipedia.org/wiki/Computer-aided_design}{CAD} to fabricate components of optical microscopes and laboratory automation controls. 
	\end{achievements}

\header{Skills}
\begin{achievements}

\item \textbf{Computing Languages:} Matlab, Python, C/C++, Fortran, Java, \LaTeX, Shell, Perl, Markdown, HTML, CSS mixed-language development (\href{http://cython.org/}{Cython}, \href{http://www.mathworks.com/help/matlab/ref/mex.html}{Mex}).

\item \textbf{Software Development:} \href{http://en.wikipedia.org/wiki/Software_configuration_management}{SCM} (\href{http://subversion.apache.org/}{svn}, \href{http://git-scm.com/}{git}, \href{http://jenkins-ci.org/}{Jenkins}), \href{http://www.gnu.org/software/make/}{make}/\href{http://www.cmake.org/}{cmake}, \href{http://www.visualstudio.com/}{Visual Studio}/\href{http://en.wikipedia.org/wiki/MSBuild}{MSBuild}, \href{http://www.eclipse.org/}{Eclipse}, \href{http://nsis.sourceforge.net/Main_Page}{NSIS}. 

\item \textbf{Compilers/Compilation}: \href{http://gcc.gnu.org/}{GNU}, \href{https://software.intel.com/en-us/c-compilers}{Intel C/C++}, Visual Studio, \href{http://www.mingw.org/}{MinGW}, \href{http://cython.org/}{Cython}, \href{http://www.mathworks.com/help/matlab/ref/mex.html}{Mex}, \href{http://ant.apache.org/}{Ant}.

\item \textbf{High-Performance Computing:} \href{http://en.wikipedia.org/wiki/Beowulf_cluster}{Linux/Unix cluster administration/computing}, parallel computation (\href{http://en.wikipedia.org/wiki/Message_Passing_Interface}{MPI}, \href{http://en.wikipedia.org/wiki/OpenMP}{OpenMP}), \href{http://en.wikipedia.org/wiki/Streaming_SIMD_Extensions}{SSE}/\href{http://en.wikipedia.org/wiki/Advanced_Vector_Extensions}{AVX} vectorization.

\item \textbf{Cloud Computing:} \href{http://aws.amazon.com/}{Amazon Web Services}, Virtualization (\href{https://www.virtualbox.org/}{VirtualBox}, \href{http://www.vmware.com/}{VMWare}).

\item \textbf{General Computing:} Linux/Unix OS (Ubuntu, Red Hat, CentOS, Mac), Windows OS, Microsoft Office, Libre/Open Office, \href{http://www.gimp.org/}{GIMP}.

\item \textbf{Open-Source Development:} \href{http://github.com/jasonlarkin}{Github}, \href{http://projects.ivec.org/gulp/}{GULP}, \href{http://lammps.sandia.gov/}{LAMMPS}, \href{http://www.ros.org/}{ROS}, \href{http://arxiv.org/find/physics/1/au:+Larkin_J/0/1/0/all/0/1}{arXiv}.

\item \textbf{Modeling:} atomistic simulation, quantum chemistry, nanoscale transport, statistical and nonlinear analysis, computational fluid dynamics. 

\item \textbf{Hardware:} optics/lasers, DI/DO AI/AO interfaces, simple automation, machining, circuitry, simple robotics control.
\end{achievements}

\header{Projects}
\begin{achievements}
\item \textbf{\href{http://ntpl.me.cmu.edu/research.html}{Quantum Mechanics-Driven Prediction of Nanostructure Thermal Conductivity}:}
served as investigator under the 
\href{http://www.wpafb.af.mil/afrl/afosr/}{AFOSR} with collaborators at Carnegie Mellon and University of Pittsburgh, performing 
calculations on the \href{http://www.hpcmo.hpc.mil/cms2/index.php}{DOD's HPCMP}.

\item \textbf{\href{https://github.com/jasonlarkin/disorder}{disorder}:} a comprehensive repository of open-source code and data from my PhD thesis, hosted on \href{http://github.com/jasonlarkin}{Github}.

\item \textbf{\href{https://github.com/ntpl/ntpy}{ntpy}:} created this open-source  collaborative effort between members of \href{http://ntpl.me.cmu.edu/}{NTPL} and \href{http://www.mie.utoronto.ca/labs/atoms/}{University of Toronto}.

\item \textbf{\href{http://projects.ivec.org/gulp/}{GULP}:} international collaboration with \href{http://nanochemistry.curtin.edu.au/people/staff.cfm/J.Gale}{Julian Gale} at the 
\href{http://nanochemistry.curtin.edu.au/}{Nanochemistry Research Institute} at \href{http://www.curtin.edu.au/}{Curtin University}.

\item \textbf{\href{http://jasonlarkin.github.io/projects-ms.html}{Statistics of Free-Surface Turbulence}:} international collaboration with \href{http://perso.ens-lyon.fr/alain.pumir/Pumir_webpage.html}{Alain Pumir} at \href{http://www.ens-lyon.eu/annuaire/m-pumir-alain-83656.kjsp?RH=ZYZYZYZYZYZYZYZYZYZYZY}{ENS Lyon}, France.

\end{achievements}

\header{\href{http://jasonlarkin.github.io/pub.html}{Publications} (selected, 11 total)}
\begin{achievements}
\item "Origin of the Exceptionally Low Thermal Conductivity of Fullerene Derivative  PCBM Films", 
\href{http://jasonlarkin.github.io/projects-phd-pcbm.html}{(in progress).}
\item "Decorrelating a Compressible Turbulent Flow: an Experiment", \href{http://pre.aps.org/abstract/PRE/v82/i1/e016301}{Physical Review E 82, 016301 (2010).}
\end{achievements}

\header{\href{http://jasonlarkin.github.io/pres.html}{Presentations} (selected, 15 total)}
\begin{achievements}
\item "Evaluation of the Virtual Crystal Approximation for Predicting Thermal Conductivity", J.M. Larkin (speaker), A.J.H.
   McGaughey, \href{http://www.mrs.org/spring2013/}{2013 MRS Spring Meeting} San Francisco, CA.
\item "The Generalized Fractal Dimensions of a 2-D Compressible Turbulence", J. Larkin (speaker), M. Bandi, W. Goldburg, \href{http://meetings.aps.org/Meeting/MAR08/Content/1017}{2008 American Physical Society March Meeting} New Orleans, LA.
\end{achievements}


\header{Honors}
\begin{achievements}
\item \href{http://www.asmeconferences.org/HT2012/}{2012 ASME MHNMT International Summer Heat Transfer Conference} Top 5 Technical Paper
\item \href{http://www.cmu.edu/me/news/archive/2011/bennett-conference.html}{2011 Bennett Conference Best Presentation}
\item \href{http://www.ices.cmu.edu/newsitem.asp?NewsID=749}{2011 ICES Northrop-Gruman Fellow}
\item 2007-2009 NSF Graduate Student Research Grant University of Pittsburgh Department of Physics.
\end{achievements}

\header{Memberships}
\begin{achievements}
\item American Physical Society $\cdot$ American Society of Mechanical Engineers 
$\cdot$ Materials Research Society $\cdot$ Society of Industrial and Applied Mathematics $\cdot$ DOD High Performance Computing Modernization Program
\end{achievements}


\end{document}
